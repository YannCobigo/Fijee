\documentclass[a4paper]{book}
\usepackage{makeidx}
\usepackage{graphicx}
\usepackage{multicol}
\usepackage{float}
\usepackage{listings}
\usepackage{color}
\usepackage{ifthen}
\usepackage[table]{xcolor}
\usepackage{textcomp}
\usepackage{alltt}
\usepackage{ifpdf}
\ifpdf
\usepackage[pdftex,
            pagebackref=true,
            colorlinks=true,
            linkcolor=blue,
            unicode
           ]{hyperref}
\else
\usepackage[ps2pdf,
            pagebackref=true,
            colorlinks=true,
            linkcolor=blue,
            unicode
           ]{hyperref}
\usepackage{pspicture}
\fi
\usepackage[utf8]{inputenc}
\usepackage{mathptmx}
\usepackage[scaled=.90]{helvet}
\usepackage{courier}
\usepackage{doxygen}
\lstset{language=C++,inputencoding=utf8,basicstyle=\footnotesize,breaklines=true,breakatwhitespace=true,tabsize=8,numbers=left }

\usepackage[OT1]{fontenc}
\usepackage{lastpage}
%\usepackage{draftcopy}

%%
%% informations AQ
%%
\newcommand{\titre}{Validation/qualification code Monte-Carlo
  Tripoli4~: r\'esultats} 
\newcommand{\auteur}{\mbox{Yann Cobigo}}
\newcommand{ \Page}{\thepage/\pageref{LastPage}}
\newcommand{ \Date}{\today}
\newcommand{\numserma}{00-0000}
\newcommand{\indicerapport}{A}
\newcommand{\LABORATOIRE}{CP2C}
\newcommand{\Tripoli}{$\times$}
\newcommand{\Darwin}{}
\newcommand{\Narmer}{}
\newcommand{\Minutes}{}
\newcommand{\Releve}{$\times$}
%
% titre  LaTeX :  si on decommante  les trois lignes  suivantes, il
%  faut  decommanter la ligne maketitle qq lignes suivantes.
%
\newcommand{\Nom}{\titre}
\newcommand{\AUTEUR}{\auteur}
\newcommand{\PAGE}{\Page}
\newcommand{\DATE}{\Date}
\newcommand{\NUMRAPPORT}{\numserma}
\newcommand{\INDICERAPPORT}{\indicerapport}
\newcommand{\SERVICE}{SERMA}
\newcommand{\TRIPOLI}{\Tripoli}
\newcommand{\DARWIN}{\Darwin}
\newcommand{\NARMER}{\Narmer}
\newcommand{\MINUTES}{\Minutes}
%
%%
%%
\usepackage{graphics}
%
% index information
\makeindex
\setcounter{tocdepth}{3}
\renewcommand{\footrulewidth}{0.4pt}

\begin{document}
%\maketitle
\begin{quote}
%
\begin{picture}(19.1,3.5)(0,0)
         \put(0.14,3.0){\framebox(8.0,0.5){}}
         \put(0.5,3.1){\bf \large Client~: CEA Saclay}
         \put(0.14,1.99){\framebox(8.0,1.0){}}
         \put(0.5,2.55){\bf \large Objet~: CIRCEE software overview}
         \put(0.5,2.09){\bf \large Lieu~~~: Saclay, \Date}
         \put(9,3.0) {\framebox(8.6,0.5){\bf R\'ef\'erence~: CIRCEE/TECH/001/A}}
         \put(9,2.52) {\framebox(0.3,0.3){}}
         \put(9.,2.57){\Minutes}
         \put(9.5,2.52){\bf Minutes de la r\'eunion}
         \put(9,2.06) {\framebox(0.3,0.3){}}
         \put(9.,2.11){\Releve}
         \put(9.5,2.06){\bf Relev\'e des d\'ecisions et des actions}
\end{picture}
%
%\vspace*{2cm}
%
\begin{center}
\textsc{\textbf{\LARGE Prestation pour la r\'e\'ecriture du code CIRCEE de Calculs d'iso-doses appliqu\'es au risque de Criticit\'e en Environnement Evolutif}}
\end{center}

\vspace*{2cm}

Diffusion:\\\hspace*{4cm}
\begin{tabular}{l}
V. Mass\'e (CEA)\\
A. Gagnier (CEA)\\
C. Magnaud (CEA)\\
F. Ab\'egil\'e(C-S)\\
M. Gonzal\`es(C-S)\\
\end{tabular}
\vspace*{1cm}

R\'edacteur:\\\hspace*{4cm}
\begin{tabular}{l}
Y. Cobigo (C-S)\\
\end{tabular}

\vspace*{2cm}
\begin{center}
\textsc{\textbf{\large Historique des versions}}
\end{center}

\vspace*{1cm}
\begin{center}
  \begin{tabular}{|c|c|c|c|} \hline
    Date & Indice & R\'edacteur de la version & Objet des modifications \\ \hline
    12/11/2008 & A & Y. Cobigo & Cr\'eation de la premi\`ere version \\ \hline
  \end{tabular}
\end{center}

\newpage

\vspace*{3cm}
\hypersetup{pageanchor=false}
\pagenumbering{roman}
\tableofcontents
\clearemptydoublepage
\pagenumbering{arabic}
\hypersetup{pageanchor=true}


%--------------------- here copy the content ---------------------%
%-----------------------------------------------------------------%


\end{quote}
\end{document}
